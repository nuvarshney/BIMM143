% Options for packages loaded elsewhere
\PassOptionsToPackage{unicode}{hyperref}
\PassOptionsToPackage{hyphens}{url}
\PassOptionsToPackage{dvipsnames,svgnames,x11names}{xcolor}
%
\documentclass[
  letterpaper,
  DIV=11,
  numbers=noendperiod]{scrartcl}

\usepackage{amsmath,amssymb}
\usepackage{iftex}
\ifPDFTeX
  \usepackage[T1]{fontenc}
  \usepackage[utf8]{inputenc}
  \usepackage{textcomp} % provide euro and other symbols
\else % if luatex or xetex
  \usepackage{unicode-math}
  \defaultfontfeatures{Scale=MatchLowercase}
  \defaultfontfeatures[\rmfamily]{Ligatures=TeX,Scale=1}
\fi
\usepackage{lmodern}
\ifPDFTeX\else  
    % xetex/luatex font selection
\fi
% Use upquote if available, for straight quotes in verbatim environments
\IfFileExists{upquote.sty}{\usepackage{upquote}}{}
\IfFileExists{microtype.sty}{% use microtype if available
  \usepackage[]{microtype}
  \UseMicrotypeSet[protrusion]{basicmath} % disable protrusion for tt fonts
}{}
\makeatletter
\@ifundefined{KOMAClassName}{% if non-KOMA class
  \IfFileExists{parskip.sty}{%
    \usepackage{parskip}
  }{% else
    \setlength{\parindent}{0pt}
    \setlength{\parskip}{6pt plus 2pt minus 1pt}}
}{% if KOMA class
  \KOMAoptions{parskip=half}}
\makeatother
\usepackage{xcolor}
\setlength{\emergencystretch}{3em} % prevent overfull lines
\setcounter{secnumdepth}{-\maxdimen} % remove section numbering
% Make \paragraph and \subparagraph free-standing
\ifx\paragraph\undefined\else
  \let\oldparagraph\paragraph
  \renewcommand{\paragraph}[1]{\oldparagraph{#1}\mbox{}}
\fi
\ifx\subparagraph\undefined\else
  \let\oldsubparagraph\subparagraph
  \renewcommand{\subparagraph}[1]{\oldsubparagraph{#1}\mbox{}}
\fi

\usepackage{color}
\usepackage{fancyvrb}
\newcommand{\VerbBar}{|}
\newcommand{\VERB}{\Verb[commandchars=\\\{\}]}
\DefineVerbatimEnvironment{Highlighting}{Verbatim}{commandchars=\\\{\}}
% Add ',fontsize=\small' for more characters per line
\usepackage{framed}
\definecolor{shadecolor}{RGB}{241,243,245}
\newenvironment{Shaded}{\begin{snugshade}}{\end{snugshade}}
\newcommand{\AlertTok}[1]{\textcolor[rgb]{0.68,0.00,0.00}{#1}}
\newcommand{\AnnotationTok}[1]{\textcolor[rgb]{0.37,0.37,0.37}{#1}}
\newcommand{\AttributeTok}[1]{\textcolor[rgb]{0.40,0.45,0.13}{#1}}
\newcommand{\BaseNTok}[1]{\textcolor[rgb]{0.68,0.00,0.00}{#1}}
\newcommand{\BuiltInTok}[1]{\textcolor[rgb]{0.00,0.23,0.31}{#1}}
\newcommand{\CharTok}[1]{\textcolor[rgb]{0.13,0.47,0.30}{#1}}
\newcommand{\CommentTok}[1]{\textcolor[rgb]{0.37,0.37,0.37}{#1}}
\newcommand{\CommentVarTok}[1]{\textcolor[rgb]{0.37,0.37,0.37}{\textit{#1}}}
\newcommand{\ConstantTok}[1]{\textcolor[rgb]{0.56,0.35,0.01}{#1}}
\newcommand{\ControlFlowTok}[1]{\textcolor[rgb]{0.00,0.23,0.31}{#1}}
\newcommand{\DataTypeTok}[1]{\textcolor[rgb]{0.68,0.00,0.00}{#1}}
\newcommand{\DecValTok}[1]{\textcolor[rgb]{0.68,0.00,0.00}{#1}}
\newcommand{\DocumentationTok}[1]{\textcolor[rgb]{0.37,0.37,0.37}{\textit{#1}}}
\newcommand{\ErrorTok}[1]{\textcolor[rgb]{0.68,0.00,0.00}{#1}}
\newcommand{\ExtensionTok}[1]{\textcolor[rgb]{0.00,0.23,0.31}{#1}}
\newcommand{\FloatTok}[1]{\textcolor[rgb]{0.68,0.00,0.00}{#1}}
\newcommand{\FunctionTok}[1]{\textcolor[rgb]{0.28,0.35,0.67}{#1}}
\newcommand{\ImportTok}[1]{\textcolor[rgb]{0.00,0.46,0.62}{#1}}
\newcommand{\InformationTok}[1]{\textcolor[rgb]{0.37,0.37,0.37}{#1}}
\newcommand{\KeywordTok}[1]{\textcolor[rgb]{0.00,0.23,0.31}{#1}}
\newcommand{\NormalTok}[1]{\textcolor[rgb]{0.00,0.23,0.31}{#1}}
\newcommand{\OperatorTok}[1]{\textcolor[rgb]{0.37,0.37,0.37}{#1}}
\newcommand{\OtherTok}[1]{\textcolor[rgb]{0.00,0.23,0.31}{#1}}
\newcommand{\PreprocessorTok}[1]{\textcolor[rgb]{0.68,0.00,0.00}{#1}}
\newcommand{\RegionMarkerTok}[1]{\textcolor[rgb]{0.00,0.23,0.31}{#1}}
\newcommand{\SpecialCharTok}[1]{\textcolor[rgb]{0.37,0.37,0.37}{#1}}
\newcommand{\SpecialStringTok}[1]{\textcolor[rgb]{0.13,0.47,0.30}{#1}}
\newcommand{\StringTok}[1]{\textcolor[rgb]{0.13,0.47,0.30}{#1}}
\newcommand{\VariableTok}[1]{\textcolor[rgb]{0.07,0.07,0.07}{#1}}
\newcommand{\VerbatimStringTok}[1]{\textcolor[rgb]{0.13,0.47,0.30}{#1}}
\newcommand{\WarningTok}[1]{\textcolor[rgb]{0.37,0.37,0.37}{\textit{#1}}}

\providecommand{\tightlist}{%
  \setlength{\itemsep}{0pt}\setlength{\parskip}{0pt}}\usepackage{longtable,booktabs,array}
\usepackage{calc} % for calculating minipage widths
% Correct order of tables after \paragraph or \subparagraph
\usepackage{etoolbox}
\makeatletter
\patchcmd\longtable{\par}{\if@noskipsec\mbox{}\fi\par}{}{}
\makeatother
% Allow footnotes in longtable head/foot
\IfFileExists{footnotehyper.sty}{\usepackage{footnotehyper}}{\usepackage{footnote}}
\makesavenoteenv{longtable}
\usepackage{graphicx}
\makeatletter
\def\maxwidth{\ifdim\Gin@nat@width>\linewidth\linewidth\else\Gin@nat@width\fi}
\def\maxheight{\ifdim\Gin@nat@height>\textheight\textheight\else\Gin@nat@height\fi}
\makeatother
% Scale images if necessary, so that they will not overflow the page
% margins by default, and it is still possible to overwrite the defaults
% using explicit options in \includegraphics[width, height, ...]{}
\setkeys{Gin}{width=\maxwidth,height=\maxheight,keepaspectratio}
% Set default figure placement to htbp
\makeatletter
\def\fps@figure{htbp}
\makeatother

\KOMAoption{captions}{tableheading}
\makeatletter
\makeatother
\makeatletter
\makeatother
\makeatletter
\@ifpackageloaded{caption}{}{\usepackage{caption}}
\AtBeginDocument{%
\ifdefined\contentsname
  \renewcommand*\contentsname{Table of contents}
\else
  \newcommand\contentsname{Table of contents}
\fi
\ifdefined\listfigurename
  \renewcommand*\listfigurename{List of Figures}
\else
  \newcommand\listfigurename{List of Figures}
\fi
\ifdefined\listtablename
  \renewcommand*\listtablename{List of Tables}
\else
  \newcommand\listtablename{List of Tables}
\fi
\ifdefined\figurename
  \renewcommand*\figurename{Figure}
\else
  \newcommand\figurename{Figure}
\fi
\ifdefined\tablename
  \renewcommand*\tablename{Table}
\else
  \newcommand\tablename{Table}
\fi
}
\@ifpackageloaded{float}{}{\usepackage{float}}
\floatstyle{ruled}
\@ifundefined{c@chapter}{\newfloat{codelisting}{h}{lop}}{\newfloat{codelisting}{h}{lop}[chapter]}
\floatname{codelisting}{Listing}
\newcommand*\listoflistings{\listof{codelisting}{List of Listings}}
\makeatother
\makeatletter
\@ifpackageloaded{caption}{}{\usepackage{caption}}
\@ifpackageloaded{subcaption}{}{\usepackage{subcaption}}
\makeatother
\makeatletter
\@ifpackageloaded{tcolorbox}{}{\usepackage[skins,breakable]{tcolorbox}}
\makeatother
\makeatletter
\@ifundefined{shadecolor}{\definecolor{shadecolor}{rgb}{.97, .97, .97}}
\makeatother
\makeatletter
\makeatother
\makeatletter
\makeatother
\ifLuaTeX
  \usepackage{selnolig}  % disable illegal ligatures
\fi
\IfFileExists{bookmark.sty}{\usepackage{bookmark}}{\usepackage{hyperref}}
\IfFileExists{xurl.sty}{\usepackage{xurl}}{} % add URL line breaks if available
\urlstyle{same} % disable monospaced font for URLs
\hypersetup{
  pdftitle={Class05: Data Vis with ggplot},
  pdfauthor={Nundini Varshney (PID: A16867985)},
  colorlinks=true,
  linkcolor={blue},
  filecolor={Maroon},
  citecolor={Blue},
  urlcolor={Blue},
  pdfcreator={LaTeX via pandoc}}

\title{Class05: Data Vis with ggplot}
\author{Nundini Varshney (PID: A16867985)}
\date{}

\begin{document}
\maketitle
\ifdefined\Shaded\renewenvironment{Shaded}{\begin{tcolorbox}[frame hidden, sharp corners, borderline west={3pt}{0pt}{shadecolor}, interior hidden, boxrule=0pt, breakable, enhanced]}{\end{tcolorbox}}\fi

\hypertarget{graphics-systems-in-r}{%
\section{Graphics systems in R}\label{graphics-systems-in-r}}

There are many graphics systems in R for making plots and figures.

We have already played a little with \textbf{``base R''} graphics and
the \texttt{plot()} function.

Today we will start learning about a popular graphics package called
\texttt{ggplot2()}.

This is an add on package - i.e.~we need to install it. I install it
(like I install any package) with the \texttt{install.packages()}
function.

\begin{Shaded}
\begin{Highlighting}[]
\FunctionTok{plot}\NormalTok{(cars)}
\end{Highlighting}
\end{Shaded}

\begin{figure}[H]

{\centering \includegraphics{Class05_files/figure-pdf/unnamed-chunk-1-1.pdf}

}

\end{figure}

Before I can use the functions from a package I have to load up the
package from my ``library''. We use the \texttt{library(ggplot2)}
command to load it up.

\begin{Shaded}
\begin{Highlighting}[]
\FunctionTok{library}\NormalTok{(ggplot2)}
\FunctionTok{ggplot}\NormalTok{(cars)}
\end{Highlighting}
\end{Shaded}

\begin{figure}[H]

{\centering \includegraphics{Class05_files/figure-pdf/unnamed-chunk-2-1.pdf}

}

\end{figure}

Every ggplot is made up of at least 3 things: - data (the numbers etc.
that will go into your plot) - aes (how the columns of data map to the
plot aesthetics) - geoms (how the plot actually looks, points, bars,
lines, etc.)

\begin{Shaded}
\begin{Highlighting}[]
\FunctionTok{ggplot}\NormalTok{(cars) }\SpecialCharTok{+}
  \FunctionTok{aes}\NormalTok{(}\AttributeTok{x=}\NormalTok{speed, }\AttributeTok{y=}\NormalTok{dist) }\SpecialCharTok{+}
  \FunctionTok{geom\_point}\NormalTok{()}
\end{Highlighting}
\end{Shaded}

\begin{figure}[H]

{\centering \includegraphics{Class05_files/figure-pdf/unnamed-chunk-3-1.pdf}

}

\end{figure}

For simple plots ggplot is more verbose - it takes more code - than base
R plot.

Add some more layers to our ggplot:

\begin{Shaded}
\begin{Highlighting}[]
\FunctionTok{ggplot}\NormalTok{(cars) }\SpecialCharTok{+}
  \FunctionTok{aes}\NormalTok{(}\AttributeTok{x=}\NormalTok{speed, }\AttributeTok{y=}\NormalTok{dist) }\SpecialCharTok{+}
  \FunctionTok{geom\_point}\NormalTok{() }\SpecialCharTok{+} 
  \FunctionTok{geom\_smooth}\NormalTok{() }\SpecialCharTok{+} 
  \FunctionTok{labs}\NormalTok{(}\AttributeTok{title=}\StringTok{"Stopping distance of old cars"}\NormalTok{,}
       \AttributeTok{subtitle =} \StringTok{"A silly example plot"}\NormalTok{) }\SpecialCharTok{+}
  \FunctionTok{theme\_bw}\NormalTok{()}
\end{Highlighting}
\end{Shaded}

\begin{verbatim}
`geom_smooth()` using method = 'loess' and formula = 'y ~ x'
\end{verbatim}

\begin{figure}[H]

{\centering \includegraphics{Class05_files/figure-pdf/unnamed-chunk-4-1.pdf}

}

\end{figure}

\begin{Shaded}
\begin{Highlighting}[]
\NormalTok{url }\OtherTok{\textless{}{-}} \StringTok{"https://bioboot.github.io/bimm143\_S20/class{-}material/up\_down\_expression.txt"}
\NormalTok{genes }\OtherTok{\textless{}{-}} \FunctionTok{read.delim}\NormalTok{(url)}
\FunctionTok{head}\NormalTok{(genes)}
\end{Highlighting}
\end{Shaded}

\begin{verbatim}
        Gene Condition1 Condition2      State
1      A4GNT -3.6808610 -3.4401355 unchanging
2       AAAS  4.5479580  4.3864126 unchanging
3      AASDH  3.7190695  3.4787276 unchanging
4       AATF  5.0784720  5.0151916 unchanging
5       AATK  0.4711421  0.5598642 unchanging
6 AB015752.4 -3.6808610 -3.5921390 unchanging
\end{verbatim}

Q. Use the nrow() function to find out how many genes are in this
dataset. What is your answer?

\begin{Shaded}
\begin{Highlighting}[]
\FunctionTok{nrow}\NormalTok{(genes)}
\end{Highlighting}
\end{Shaded}

\begin{verbatim}
[1] 5196
\end{verbatim}

Q. Use the colnames() function and the ncol() function on the genes data
frame to find out what the column names are (we will need these later)
and how many columns there are. How many columns did you find?

\begin{Shaded}
\begin{Highlighting}[]
\FunctionTok{colnames}\NormalTok{(genes)}
\end{Highlighting}
\end{Shaded}

\begin{verbatim}
[1] "Gene"       "Condition1" "Condition2" "State"     
\end{verbatim}

\begin{Shaded}
\begin{Highlighting}[]
\FunctionTok{ncol}\NormalTok{(genes)}
\end{Highlighting}
\end{Shaded}

\begin{verbatim}
[1] 4
\end{verbatim}

Q. Use the table() function on the State column of this data.frame to
find out how many `up' regulated genes there are. What is your answer?

\begin{Shaded}
\begin{Highlighting}[]
\FunctionTok{table}\NormalTok{(genes[,}\StringTok{"State"}\NormalTok{])}
\end{Highlighting}
\end{Shaded}

\begin{verbatim}

      down unchanging         up 
        72       4997        127 
\end{verbatim}

Q. Using your values above and 2 significant figures. What fraction of
total genes is up-regulated in this dataset?

\begin{Shaded}
\begin{Highlighting}[]
\FunctionTok{round}\NormalTok{(}\FunctionTok{table}\NormalTok{(genes}\SpecialCharTok{$}\NormalTok{State)}\SpecialCharTok{/}\FunctionTok{nrow}\NormalTok{(genes) }\SpecialCharTok{*} \DecValTok{100}\NormalTok{, }\DecValTok{2}\NormalTok{)}
\end{Highlighting}
\end{Shaded}

\begin{verbatim}

      down unchanging         up 
      1.39      96.17       2.44 
\end{verbatim}

Q. Complete the code below to produce the following plot

\begin{Shaded}
\begin{Highlighting}[]
\FunctionTok{ggplot}\NormalTok{(genes) }\SpecialCharTok{+} 
    \FunctionTok{aes}\NormalTok{(}\AttributeTok{x=}\NormalTok{Condition1, }\AttributeTok{y=}\NormalTok{Condition2) }\SpecialCharTok{+}
    \FunctionTok{geom\_point}\NormalTok{()}
\end{Highlighting}
\end{Shaded}

\begin{figure}[H]

{\centering \includegraphics{Class05_files/figure-pdf/unnamed-chunk-10-1.pdf}

}

\end{figure}

\begin{Shaded}
\begin{Highlighting}[]
\NormalTok{p }\OtherTok{\textless{}{-}} \FunctionTok{ggplot}\NormalTok{(genes) }\SpecialCharTok{+} 
    \FunctionTok{aes}\NormalTok{(}\AttributeTok{x=}\NormalTok{Condition1, }\AttributeTok{y=}\NormalTok{Condition2, }\AttributeTok{col=}\NormalTok{State) }\SpecialCharTok{+}
    \FunctionTok{geom\_point}\NormalTok{()}
\NormalTok{p }\SpecialCharTok{+} \FunctionTok{scale\_colour\_manual}\NormalTok{( }\AttributeTok{values=}\FunctionTok{c}\NormalTok{(}\StringTok{"blue"}\NormalTok{,}\StringTok{"gray"}\NormalTok{,}\StringTok{"red"}\NormalTok{) )}
\end{Highlighting}
\end{Shaded}

\begin{figure}[H]

{\centering \includegraphics{Class05_files/figure-pdf/unnamed-chunk-11-1.pdf}

}

\end{figure}

Q. Nice, now add some plot annotations to the p object with the labs()
function so your plot looks like the following:

\begin{Shaded}
\begin{Highlighting}[]
\NormalTok{p }\SpecialCharTok{+} \FunctionTok{scale\_colour\_manual}\NormalTok{(}\AttributeTok{values=}\FunctionTok{c}\NormalTok{(}\StringTok{"blue"}\NormalTok{,}\StringTok{"gray"}\NormalTok{,}\StringTok{"red"}\NormalTok{)) }\SpecialCharTok{+}
    \FunctionTok{labs}\NormalTok{(}\AttributeTok{title=}\StringTok{"Gene Expresion Changes Upon Drug Treatment"}\NormalTok{,}
         \AttributeTok{x=}\StringTok{"Control (no drug) "}\NormalTok{,}
         \AttributeTok{y=}\StringTok{"Drug Treatment"}\NormalTok{)}
\end{Highlighting}
\end{Shaded}

\begin{figure}[H]

{\centering \includegraphics{Class05_files/figure-pdf/unnamed-chunk-12-1.pdf}

}

\end{figure}

\begin{Shaded}
\begin{Highlighting}[]
\NormalTok{url }\OtherTok{\textless{}{-}} \StringTok{"https://raw.githubusercontent.com/jennybc/gapminder/master/inst/extdata/gapminder.tsv"}
\NormalTok{gapminder }\OtherTok{\textless{}{-}} \FunctionTok{read.delim}\NormalTok{(url)}
\end{Highlighting}
\end{Shaded}

\begin{Shaded}
\begin{Highlighting}[]
\FunctionTok{library}\NormalTok{(dplyr)}
\end{Highlighting}
\end{Shaded}

\begin{verbatim}

Attaching package: 'dplyr'
\end{verbatim}

\begin{verbatim}
The following objects are masked from 'package:stats':

    filter, lag
\end{verbatim}

\begin{verbatim}
The following objects are masked from 'package:base':

    intersect, setdiff, setequal, union
\end{verbatim}

\begin{Shaded}
\begin{Highlighting}[]
\NormalTok{gapminder\_2007 }\OtherTok{\textless{}{-}}\NormalTok{ gapminder }\SpecialCharTok{\%\textgreater{}\%} \FunctionTok{filter}\NormalTok{(year}\SpecialCharTok{==}\DecValTok{2007}\NormalTok{)}
\end{Highlighting}
\end{Shaded}

Q. Complete the code below to produce a first basic scater plot of this
gapminder\_2007 dataset:

\begin{Shaded}
\begin{Highlighting}[]
\FunctionTok{ggplot}\NormalTok{(gapminder\_2007) }\SpecialCharTok{+}
  \FunctionTok{aes}\NormalTok{(}\AttributeTok{x=}\NormalTok{gdpPercap, }\AttributeTok{y=}\NormalTok{lifeExp, }\AttributeTok{color=}\NormalTok{continent, }\AttributeTok{size=}\NormalTok{pop) }\SpecialCharTok{+}
  \FunctionTok{geom\_point}\NormalTok{(}\AttributeTok{alpha=}\FloatTok{0.5}\NormalTok{)}
\end{Highlighting}
\end{Shaded}

\begin{figure}[H]

{\centering \includegraphics{Class05_files/figure-pdf/unnamed-chunk-15-1.pdf}

}

\end{figure}

By contrast, let's see how the plot looks like if we color the points by
the numeric variable population pop:

\begin{Shaded}
\begin{Highlighting}[]
\FunctionTok{ggplot}\NormalTok{(gapminder\_2007) }\SpecialCharTok{+} 
  \FunctionTok{aes}\NormalTok{(}\AttributeTok{x =}\NormalTok{ gdpPercap, }\AttributeTok{y =}\NormalTok{ lifeExp, }\AttributeTok{color =}\NormalTok{ pop) }\SpecialCharTok{+}
  \FunctionTok{geom\_point}\NormalTok{(}\AttributeTok{alpha=}\FloatTok{0.8}\NormalTok{)}
\end{Highlighting}
\end{Shaded}

\begin{figure}[H]

{\centering \includegraphics{Class05_files/figure-pdf/unnamed-chunk-16-1.pdf}

}

\end{figure}

Adjusting point size:

\begin{Shaded}
\begin{Highlighting}[]
\FunctionTok{ggplot}\NormalTok{(gapminder\_2007) }\SpecialCharTok{+} 
  \FunctionTok{aes}\NormalTok{(}\AttributeTok{x =}\NormalTok{ gdpPercap, }\AttributeTok{y =}\NormalTok{ lifeExp, }\AttributeTok{size =}\NormalTok{ pop) }\SpecialCharTok{+}
  \FunctionTok{geom\_point}\NormalTok{(}\AttributeTok{alpha=}\FloatTok{0.5}\NormalTok{)}
\end{Highlighting}
\end{Shaded}

\begin{figure}[H]

{\centering \includegraphics{Class05_files/figure-pdf/unnamed-chunk-17-1.pdf}

}

\end{figure}

\begin{Shaded}
\begin{Highlighting}[]
\FunctionTok{ggplot}\NormalTok{(gapminder\_2007) }\SpecialCharTok{+} 
  \FunctionTok{geom\_point}\NormalTok{(}\FunctionTok{aes}\NormalTok{(}\AttributeTok{x =}\NormalTok{ gdpPercap, }\AttributeTok{y =}\NormalTok{ lifeExp,}
                 \AttributeTok{size =}\NormalTok{ pop), }\AttributeTok{alpha=}\FloatTok{0.5}\NormalTok{) }\SpecialCharTok{+} 
  \FunctionTok{scale\_size\_area}\NormalTok{(}\AttributeTok{max\_size =} \DecValTok{10}\NormalTok{)}
\end{Highlighting}
\end{Shaded}

\begin{figure}[H]

{\centering \includegraphics{Class05_files/figure-pdf/unnamed-chunk-18-1.pdf}

}

\end{figure}

Q. Can you adapt the code you have learned thus far to reproduce our
gapminder scatter plot for the year 1957? What do you notice about this
plot is it easy to compare with the one for 2007?

\begin{Shaded}
\begin{Highlighting}[]
\NormalTok{gapminder\_1957 }\OtherTok{\textless{}{-}}\NormalTok{ gapminder }\SpecialCharTok{\%\textgreater{}\%} \FunctionTok{filter}\NormalTok{(year}\SpecialCharTok{==}\DecValTok{1957}\NormalTok{)}
\FunctionTok{ggplot}\NormalTok{(gapminder\_1957) }\SpecialCharTok{+} 
  \FunctionTok{aes}\NormalTok{(}\AttributeTok{x =}\NormalTok{ gdpPercap, }\AttributeTok{y =}\NormalTok{ lifeExp, }\AttributeTok{color=}\NormalTok{continent,}
                 \AttributeTok{size =}\NormalTok{ pop) }\SpecialCharTok{+}
  \FunctionTok{geom\_point}\NormalTok{(}\AttributeTok{alpha=}\FloatTok{0.7}\NormalTok{) }\SpecialCharTok{+} 
  \FunctionTok{scale\_size\_area}\NormalTok{(}\AttributeTok{max\_size =} \DecValTok{15}\NormalTok{) }
\end{Highlighting}
\end{Shaded}

\begin{figure}[H]

{\centering \includegraphics{Class05_files/figure-pdf/unnamed-chunk-19-1.pdf}

}

\end{figure}

Q. Do the same steps above but include 1957 and 2007 in your input
dataset for ggplot(). You should now include the layer
facet\_wrap(\textasciitilde year) to produce the following plot:

\begin{Shaded}
\begin{Highlighting}[]
\NormalTok{gapminder\_1957 }\OtherTok{\textless{}{-}}\NormalTok{ gapminder }\SpecialCharTok{\%\textgreater{}\%} \FunctionTok{filter}\NormalTok{(year}\SpecialCharTok{==}\DecValTok{1957} \SpecialCharTok{|}\NormalTok{ year}\SpecialCharTok{==}\DecValTok{2007}\NormalTok{)}
\FunctionTok{ggplot}\NormalTok{(gapminder\_1957) }\SpecialCharTok{+} 
  \FunctionTok{aes}\NormalTok{(}\AttributeTok{x =}\NormalTok{ gdpPercap, }\AttributeTok{y =}\NormalTok{ lifeExp, }\AttributeTok{color=}\NormalTok{continent,}
                 \AttributeTok{size =}\NormalTok{ pop) }\SpecialCharTok{+}
  \FunctionTok{geom\_point}\NormalTok{(}\AttributeTok{alpha=}\FloatTok{0.7}\NormalTok{) }\SpecialCharTok{+} 
  \FunctionTok{scale\_size\_area}\NormalTok{(}\AttributeTok{max\_size =} \DecValTok{15}\NormalTok{) }\SpecialCharTok{+}
  \FunctionTok{facet\_wrap}\NormalTok{(}\SpecialCharTok{\textasciitilde{}}\NormalTok{year)}
\end{Highlighting}
\end{Shaded}

\begin{figure}[H]

{\centering \includegraphics{Class05_files/figure-pdf/unnamed-chunk-20-1.pdf}

}

\end{figure}



\end{document}
