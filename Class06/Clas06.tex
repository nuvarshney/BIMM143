% Options for packages loaded elsewhere
\PassOptionsToPackage{unicode}{hyperref}
\PassOptionsToPackage{hyphens}{url}
\PassOptionsToPackage{dvipsnames,svgnames,x11names}{xcolor}
%
\documentclass[
  letterpaper,
  DIV=11,
  numbers=noendperiod]{scrartcl}

\usepackage{amsmath,amssymb}
\usepackage{iftex}
\ifPDFTeX
  \usepackage[T1]{fontenc}
  \usepackage[utf8]{inputenc}
  \usepackage{textcomp} % provide euro and other symbols
\else % if luatex or xetex
  \usepackage{unicode-math}
  \defaultfontfeatures{Scale=MatchLowercase}
  \defaultfontfeatures[\rmfamily]{Ligatures=TeX,Scale=1}
\fi
\usepackage{lmodern}
\ifPDFTeX\else  
    % xetex/luatex font selection
\fi
% Use upquote if available, for straight quotes in verbatim environments
\IfFileExists{upquote.sty}{\usepackage{upquote}}{}
\IfFileExists{microtype.sty}{% use microtype if available
  \usepackage[]{microtype}
  \UseMicrotypeSet[protrusion]{basicmath} % disable protrusion for tt fonts
}{}
\makeatletter
\@ifundefined{KOMAClassName}{% if non-KOMA class
  \IfFileExists{parskip.sty}{%
    \usepackage{parskip}
  }{% else
    \setlength{\parindent}{0pt}
    \setlength{\parskip}{6pt plus 2pt minus 1pt}}
}{% if KOMA class
  \KOMAoptions{parskip=half}}
\makeatother
\usepackage{xcolor}
\setlength{\emergencystretch}{3em} % prevent overfull lines
\setcounter{secnumdepth}{-\maxdimen} % remove section numbering
% Make \paragraph and \subparagraph free-standing
\ifx\paragraph\undefined\else
  \let\oldparagraph\paragraph
  \renewcommand{\paragraph}[1]{\oldparagraph{#1}\mbox{}}
\fi
\ifx\subparagraph\undefined\else
  \let\oldsubparagraph\subparagraph
  \renewcommand{\subparagraph}[1]{\oldsubparagraph{#1}\mbox{}}
\fi

\usepackage{color}
\usepackage{fancyvrb}
\newcommand{\VerbBar}{|}
\newcommand{\VERB}{\Verb[commandchars=\\\{\}]}
\DefineVerbatimEnvironment{Highlighting}{Verbatim}{commandchars=\\\{\}}
% Add ',fontsize=\small' for more characters per line
\usepackage{framed}
\definecolor{shadecolor}{RGB}{241,243,245}
\newenvironment{Shaded}{\begin{snugshade}}{\end{snugshade}}
\newcommand{\AlertTok}[1]{\textcolor[rgb]{0.68,0.00,0.00}{#1}}
\newcommand{\AnnotationTok}[1]{\textcolor[rgb]{0.37,0.37,0.37}{#1}}
\newcommand{\AttributeTok}[1]{\textcolor[rgb]{0.40,0.45,0.13}{#1}}
\newcommand{\BaseNTok}[1]{\textcolor[rgb]{0.68,0.00,0.00}{#1}}
\newcommand{\BuiltInTok}[1]{\textcolor[rgb]{0.00,0.23,0.31}{#1}}
\newcommand{\CharTok}[1]{\textcolor[rgb]{0.13,0.47,0.30}{#1}}
\newcommand{\CommentTok}[1]{\textcolor[rgb]{0.37,0.37,0.37}{#1}}
\newcommand{\CommentVarTok}[1]{\textcolor[rgb]{0.37,0.37,0.37}{\textit{#1}}}
\newcommand{\ConstantTok}[1]{\textcolor[rgb]{0.56,0.35,0.01}{#1}}
\newcommand{\ControlFlowTok}[1]{\textcolor[rgb]{0.00,0.23,0.31}{#1}}
\newcommand{\DataTypeTok}[1]{\textcolor[rgb]{0.68,0.00,0.00}{#1}}
\newcommand{\DecValTok}[1]{\textcolor[rgb]{0.68,0.00,0.00}{#1}}
\newcommand{\DocumentationTok}[1]{\textcolor[rgb]{0.37,0.37,0.37}{\textit{#1}}}
\newcommand{\ErrorTok}[1]{\textcolor[rgb]{0.68,0.00,0.00}{#1}}
\newcommand{\ExtensionTok}[1]{\textcolor[rgb]{0.00,0.23,0.31}{#1}}
\newcommand{\FloatTok}[1]{\textcolor[rgb]{0.68,0.00,0.00}{#1}}
\newcommand{\FunctionTok}[1]{\textcolor[rgb]{0.28,0.35,0.67}{#1}}
\newcommand{\ImportTok}[1]{\textcolor[rgb]{0.00,0.46,0.62}{#1}}
\newcommand{\InformationTok}[1]{\textcolor[rgb]{0.37,0.37,0.37}{#1}}
\newcommand{\KeywordTok}[1]{\textcolor[rgb]{0.00,0.23,0.31}{#1}}
\newcommand{\NormalTok}[1]{\textcolor[rgb]{0.00,0.23,0.31}{#1}}
\newcommand{\OperatorTok}[1]{\textcolor[rgb]{0.37,0.37,0.37}{#1}}
\newcommand{\OtherTok}[1]{\textcolor[rgb]{0.00,0.23,0.31}{#1}}
\newcommand{\PreprocessorTok}[1]{\textcolor[rgb]{0.68,0.00,0.00}{#1}}
\newcommand{\RegionMarkerTok}[1]{\textcolor[rgb]{0.00,0.23,0.31}{#1}}
\newcommand{\SpecialCharTok}[1]{\textcolor[rgb]{0.37,0.37,0.37}{#1}}
\newcommand{\SpecialStringTok}[1]{\textcolor[rgb]{0.13,0.47,0.30}{#1}}
\newcommand{\StringTok}[1]{\textcolor[rgb]{0.13,0.47,0.30}{#1}}
\newcommand{\VariableTok}[1]{\textcolor[rgb]{0.07,0.07,0.07}{#1}}
\newcommand{\VerbatimStringTok}[1]{\textcolor[rgb]{0.13,0.47,0.30}{#1}}
\newcommand{\WarningTok}[1]{\textcolor[rgb]{0.37,0.37,0.37}{\textit{#1}}}

\providecommand{\tightlist}{%
  \setlength{\itemsep}{0pt}\setlength{\parskip}{0pt}}\usepackage{longtable,booktabs,array}
\usepackage{calc} % for calculating minipage widths
% Correct order of tables after \paragraph or \subparagraph
\usepackage{etoolbox}
\makeatletter
\patchcmd\longtable{\par}{\if@noskipsec\mbox{}\fi\par}{}{}
\makeatother
% Allow footnotes in longtable head/foot
\IfFileExists{footnotehyper.sty}{\usepackage{footnotehyper}}{\usepackage{footnote}}
\makesavenoteenv{longtable}
\usepackage{graphicx}
\makeatletter
\def\maxwidth{\ifdim\Gin@nat@width>\linewidth\linewidth\else\Gin@nat@width\fi}
\def\maxheight{\ifdim\Gin@nat@height>\textheight\textheight\else\Gin@nat@height\fi}
\makeatother
% Scale images if necessary, so that they will not overflow the page
% margins by default, and it is still possible to overwrite the defaults
% using explicit options in \includegraphics[width, height, ...]{}
\setkeys{Gin}{width=\maxwidth,height=\maxheight,keepaspectratio}
% Set default figure placement to htbp
\makeatletter
\def\fps@figure{htbp}
\makeatother

\KOMAoption{captions}{tableheading}
\makeatletter
\makeatother
\makeatletter
\makeatother
\makeatletter
\@ifpackageloaded{caption}{}{\usepackage{caption}}
\AtBeginDocument{%
\ifdefined\contentsname
  \renewcommand*\contentsname{Table of contents}
\else
  \newcommand\contentsname{Table of contents}
\fi
\ifdefined\listfigurename
  \renewcommand*\listfigurename{List of Figures}
\else
  \newcommand\listfigurename{List of Figures}
\fi
\ifdefined\listtablename
  \renewcommand*\listtablename{List of Tables}
\else
  \newcommand\listtablename{List of Tables}
\fi
\ifdefined\figurename
  \renewcommand*\figurename{Figure}
\else
  \newcommand\figurename{Figure}
\fi
\ifdefined\tablename
  \renewcommand*\tablename{Table}
\else
  \newcommand\tablename{Table}
\fi
}
\@ifpackageloaded{float}{}{\usepackage{float}}
\floatstyle{ruled}
\@ifundefined{c@chapter}{\newfloat{codelisting}{h}{lop}}{\newfloat{codelisting}{h}{lop}[chapter]}
\floatname{codelisting}{Listing}
\newcommand*\listoflistings{\listof{codelisting}{List of Listings}}
\makeatother
\makeatletter
\@ifpackageloaded{caption}{}{\usepackage{caption}}
\@ifpackageloaded{subcaption}{}{\usepackage{subcaption}}
\makeatother
\makeatletter
\@ifpackageloaded{tcolorbox}{}{\usepackage[skins,breakable]{tcolorbox}}
\makeatother
\makeatletter
\@ifundefined{shadecolor}{\definecolor{shadecolor}{rgb}{.97, .97, .97}}
\makeatother
\makeatletter
\makeatother
\makeatletter
\makeatother
\ifLuaTeX
  \usepackage{selnolig}  % disable illegal ligatures
\fi
\IfFileExists{bookmark.sty}{\usepackage{bookmark}}{\usepackage{hyperref}}
\IfFileExists{xurl.sty}{\usepackage{xurl}}{} % add URL line breaks if available
\urlstyle{same} % disable monospaced font for URLs
\hypersetup{
  pdftitle={Class 6: R Functions},
  pdfauthor={Nundini Varshney (PID: A16867985)},
  colorlinks=true,
  linkcolor={blue},
  filecolor={Maroon},
  citecolor={Blue},
  urlcolor={Blue},
  pdfcreator={LaTeX via pandoc}}

\title{Class 6: R Functions}
\author{Nundini Varshney (PID: A16867985)}
\date{2024-01-25}

\begin{document}
\maketitle
\ifdefined\Shaded\renewenvironment{Shaded}{\begin{tcolorbox}[enhanced, breakable, boxrule=0pt, interior hidden, borderline west={3pt}{0pt}{shadecolor}, sharp corners, frame hidden]}{\end{tcolorbox}}\fi

\hypertarget{r-functions}{%
\subsection{R Functions}\label{r-functions}}

Functions are how we get stuff done. We call functions to do everything
useful in R.

One cool thing about R is that it makes writing your own functions
comparatively easy.

All functions in R have at least three things:

\begin{itemize}
\tightlist
\item
  A \textbf{name} (we get to pick this)
\item
  One or more \textbf{input arguments} (the input to our function)
\item
  The \textbf{body} (lines of code that do the work)
\end{itemize}

\begin{Shaded}
\begin{Highlighting}[]
\CommentTok{\#/ eval: false}

\NormalTok{funname }\OtherTok{\textless{}{-}} \ControlFlowTok{function}\NormalTok{(input1, input2) \{}
\CommentTok{\# The body with R code}
\NormalTok{\}}
\end{Highlighting}
\end{Shaded}

Let's write a silly first function to add two numbers:

\begin{Shaded}
\begin{Highlighting}[]
\NormalTok{x }\OtherTok{\textless{}{-}} \DecValTok{5}
\NormalTok{y }\OtherTok{\textless{}{-}} \DecValTok{1}
\NormalTok{x }\SpecialCharTok{+}\NormalTok{ y}
\end{Highlighting}
\end{Shaded}

\begin{verbatim}
[1] 6
\end{verbatim}

\begin{Shaded}
\begin{Highlighting}[]
\NormalTok{addme }\OtherTok{\textless{}{-}} \ControlFlowTok{function}\NormalTok{(x, }\AttributeTok{y =} \DecValTok{1}\NormalTok{) \{}
\NormalTok{  x }\SpecialCharTok{+}\NormalTok{ y}
\NormalTok{\}}
\end{Highlighting}
\end{Shaded}

\begin{Shaded}
\begin{Highlighting}[]
\FunctionTok{addme}\NormalTok{(}\DecValTok{100}\NormalTok{,}\DecValTok{100}\NormalTok{)}
\end{Highlighting}
\end{Shaded}

\begin{verbatim}
[1] 200
\end{verbatim}

\begin{Shaded}
\begin{Highlighting}[]
\FunctionTok{addme}\NormalTok{(}\DecValTok{10}\NormalTok{)}
\end{Highlighting}
\end{Shaded}

\begin{verbatim}
[1] 11
\end{verbatim}

\hypertarget{lab-for-today}{%
\subsection{Lab for today}\label{lab-for-today}}

Write a function to grade student work from class.

Start with a simplified version of the problem:

\begin{Shaded}
\begin{Highlighting}[]
\CommentTok{\# Example input vectors to start with}
\NormalTok{student1 }\OtherTok{\textless{}{-}} \FunctionTok{c}\NormalTok{(}\DecValTok{100}\NormalTok{, }\DecValTok{100}\NormalTok{, }\DecValTok{100}\NormalTok{, }\DecValTok{100}\NormalTok{, }\DecValTok{100}\NormalTok{, }\DecValTok{100}\NormalTok{, }\DecValTok{100}\NormalTok{, }\DecValTok{90}\NormalTok{)}
\NormalTok{student2 }\OtherTok{\textless{}{-}} \FunctionTok{c}\NormalTok{(}\DecValTok{100}\NormalTok{, }\ConstantTok{NA}\NormalTok{, }\DecValTok{90}\NormalTok{, }\DecValTok{90}\NormalTok{, }\DecValTok{90}\NormalTok{, }\DecValTok{90}\NormalTok{, }\DecValTok{97}\NormalTok{, }\DecValTok{80}\NormalTok{)}
\NormalTok{student3 }\OtherTok{\textless{}{-}} \FunctionTok{c}\NormalTok{(}\DecValTok{90}\NormalTok{, }\ConstantTok{NA}\NormalTok{, }\ConstantTok{NA}\NormalTok{, }\ConstantTok{NA}\NormalTok{, }\ConstantTok{NA}\NormalTok{, }\ConstantTok{NA}\NormalTok{, }\ConstantTok{NA}\NormalTok{, }\ConstantTok{NA}\NormalTok{)}
\end{Highlighting}
\end{Shaded}

Let's find the average.

\begin{Shaded}
\begin{Highlighting}[]
\FunctionTok{mean}\NormalTok{(student1)}
\end{Highlighting}
\end{Shaded}

\begin{verbatim}
[1] 98.75
\end{verbatim}

\begin{Shaded}
\begin{Highlighting}[]
\FunctionTok{mean}\NormalTok{(student2, }\AttributeTok{na.rm =} \ConstantTok{TRUE}\NormalTok{)}
\end{Highlighting}
\end{Shaded}

\begin{verbatim}
[1] 91
\end{verbatim}

\begin{Shaded}
\begin{Highlighting}[]
\FunctionTok{mean}\NormalTok{(student3, }\AttributeTok{na.rm =} \ConstantTok{TRUE}\NormalTok{)}
\end{Highlighting}
\end{Shaded}

\begin{verbatim}
[1] 90
\end{verbatim}

This is not fair - there is no way student3 should have a mean of 90.

Come back to this NA problem. But things worked for \texttt{student1}.

We want to drop the lowest score before getting the \texttt{mean()}

How do I find the lowest (minimum) score?

\begin{Shaded}
\begin{Highlighting}[]
\NormalTok{student1}
\end{Highlighting}
\end{Shaded}

\begin{verbatim}
[1] 100 100 100 100 100 100 100  90
\end{verbatim}

\begin{Shaded}
\begin{Highlighting}[]
\FunctionTok{min}\NormalTok{(student1)}
\end{Highlighting}
\end{Shaded}

\begin{verbatim}
[1] 90
\end{verbatim}

I found the \texttt{which.min()} function. Maybe this is more useful?

\begin{Shaded}
\begin{Highlighting}[]
\FunctionTok{which.min}\NormalTok{(student1)}
\end{Highlighting}
\end{Shaded}

\begin{verbatim}
[1] 8
\end{verbatim}

Cool - it is the 8th element of the vector that has the lowest score.
Can I remove this one?

\begin{Shaded}
\begin{Highlighting}[]
\NormalTok{student1[ }\DecValTok{8}\NormalTok{ ]}
\end{Highlighting}
\end{Shaded}

\begin{verbatim}
[1] 90
\end{verbatim}

We can use the wee minus trick for indexing.

\begin{Shaded}
\begin{Highlighting}[]
\NormalTok{x }\OtherTok{\textless{}{-}} \DecValTok{1}\SpecialCharTok{:}\DecValTok{5}
\NormalTok{x[}\SpecialCharTok{{-}}\DecValTok{3}\NormalTok{]}
\end{Highlighting}
\end{Shaded}

\begin{verbatim}
[1] 1 2 4 5
\end{verbatim}

Now put these bits of knowledge together to make some code that
identifies and drops the lowest score (element of the input vector) and
then calculates the mean.

\begin{Shaded}
\begin{Highlighting}[]
\CommentTok{\# Find the lowest score}
\NormalTok{ind }\OtherTok{\textless{}{-}} \FunctionTok{which.min}\NormalTok{(student1)}
\CommentTok{\# remove lowest score and find the mean}
\FunctionTok{mean}\NormalTok{( student1[}\SpecialCharTok{{-}}\NormalTok{ind])}
\end{Highlighting}
\end{Shaded}

\begin{verbatim}
[1] 100
\end{verbatim}

Use a common shortcut and use \texttt{x} as my input

\begin{Shaded}
\begin{Highlighting}[]
\NormalTok{x }\OtherTok{\textless{}{-}}\NormalTok{ student1}
\FunctionTok{mean}\NormalTok{( x[ }\SpecialCharTok{{-}}\FunctionTok{which.min}\NormalTok{(x) ] )}
\end{Highlighting}
\end{Shaded}

\begin{verbatim}
[1] 100
\end{verbatim}

We still have the problem of missing values.

One idea is to replace NA values with zero.

\begin{Shaded}
\begin{Highlighting}[]
\NormalTok{y }\OtherTok{\textless{}{-}} \DecValTok{1}\SpecialCharTok{:}\DecValTok{5}
\NormalTok{y[y }\SpecialCharTok{==}\DecValTok{3}\NormalTok{] }\OtherTok{\textless{}{-}} \DecValTok{10000}
\NormalTok{y}
\end{Highlighting}
\end{Shaded}

\begin{verbatim}
[1]     1     2 10000     4     5
\end{verbatim}

Bummer, this is no good\ldots{}

\begin{Shaded}
\begin{Highlighting}[]
\NormalTok{y }\OtherTok{\textless{}{-}} \FunctionTok{c}\NormalTok{(}\DecValTok{1}\NormalTok{, }\DecValTok{2}\NormalTok{, }\ConstantTok{NA}\NormalTok{, }\DecValTok{4}\NormalTok{, }\DecValTok{5}\NormalTok{)}
\NormalTok{y }\SpecialCharTok{==} \ConstantTok{NA}
\end{Highlighting}
\end{Shaded}

\begin{verbatim}
[1] NA NA NA NA NA
\end{verbatim}

\begin{Shaded}
\begin{Highlighting}[]
\NormalTok{y}
\end{Highlighting}
\end{Shaded}

\begin{verbatim}
[1]  1  2 NA  4  5
\end{verbatim}

\begin{Shaded}
\begin{Highlighting}[]
\FunctionTok{is.na}\NormalTok{(y)}
\end{Highlighting}
\end{Shaded}

\begin{verbatim}
[1] FALSE FALSE  TRUE FALSE FALSE
\end{verbatim}

How can I remove the NA elements from the vector?

\begin{Shaded}
\begin{Highlighting}[]
\SpecialCharTok{!}\FunctionTok{c}\NormalTok{(F,F,F)}
\end{Highlighting}
\end{Shaded}

\begin{verbatim}
[1] TRUE TRUE TRUE
\end{verbatim}

\begin{Shaded}
\begin{Highlighting}[]
\CommentTok{\#y[{-}is.na(y)]}
\end{Highlighting}
\end{Shaded}

\begin{Shaded}
\begin{Highlighting}[]
\NormalTok{y[ }\SpecialCharTok{!}\FunctionTok{is.na}\NormalTok{(y) ]}
\end{Highlighting}
\end{Shaded}

\begin{verbatim}
[1] 1 2 4 5
\end{verbatim}

\begin{Shaded}
\begin{Highlighting}[]
\NormalTok{y [ }\FunctionTok{is.na}\NormalTok{(y) ] }\OtherTok{\textless{}{-}} \DecValTok{10000}
\NormalTok{y}
\end{Highlighting}
\end{Shaded}

\begin{verbatim}
[1]     1     2 10000     4     5
\end{verbatim}

OK lets solve this:

\begin{Shaded}
\begin{Highlighting}[]
\NormalTok{x }\OtherTok{\textless{}{-}}\NormalTok{ student3}

\CommentTok{\# Change NA values to Zero}
\NormalTok{x[ }\FunctionTok{is.na}\NormalTok{(x) ] }\OtherTok{\textless{}{-}} \DecValTok{0}
\CommentTok{\# Find and remove min value and get mean}
\FunctionTok{mean}\NormalTok{( x[ }\SpecialCharTok{{-}}\FunctionTok{which.min}\NormalTok{(x) ] )}
\end{Highlighting}
\end{Shaded}

\begin{verbatim}
[1] 12.85714
\end{verbatim}

Last step now that I have my working code snippet is to make my
\texttt{grade()} function.

Q1. Write a function grade() to determine an overall grade from a vector
of student homework assignment scores dropping the lowest single score.
If a student misses a homework (i.e.~has an NA value) this can be used
as a score to be potentially dropped. Your final function should be
adquately explained with code comments and be able to work on an example
class gradebook such as this one in CSV format:
``https://tinyurl.com/gradeinput'' {[}3pts{]}

\begin{Shaded}
\begin{Highlighting}[]
\NormalTok{grade }\OtherTok{\textless{}{-}} \ControlFlowTok{function}\NormalTok{(x) \{}
\NormalTok{  x[ }\FunctionTok{is.na}\NormalTok{(x) ] }\OtherTok{\textless{}{-}} \DecValTok{0}
  \CommentTok{\# Find and remove min value and get mean}
  \FunctionTok{mean}\NormalTok{( x[ }\SpecialCharTok{{-}}\FunctionTok{which.min}\NormalTok{(x) ] )}
\NormalTok{\}}
\end{Highlighting}
\end{Shaded}

\begin{Shaded}
\begin{Highlighting}[]
\FunctionTok{grade}\NormalTok{(student1)}
\end{Highlighting}
\end{Shaded}

\begin{verbatim}
[1] 100
\end{verbatim}

\begin{Shaded}
\begin{Highlighting}[]
\FunctionTok{grade}\NormalTok{(student2)}
\end{Highlighting}
\end{Shaded}

\begin{verbatim}
[1] 91
\end{verbatim}

\begin{Shaded}
\begin{Highlighting}[]
\FunctionTok{grade}\NormalTok{(student3)}
\end{Highlighting}
\end{Shaded}

\begin{verbatim}
[1] 12.85714
\end{verbatim}

Now read the online gradebook (CSV file)

\begin{Shaded}
\begin{Highlighting}[]
\NormalTok{url }\OtherTok{\textless{}{-}} \StringTok{"https://tinyurl.com/gradeinput"}
\NormalTok{gradebook }\OtherTok{\textless{}{-}} \FunctionTok{read.csv}\NormalTok{(url, }\AttributeTok{row.names =} \DecValTok{1}\NormalTok{)}

\FunctionTok{head}\NormalTok{(gradebook)}
\end{Highlighting}
\end{Shaded}

\begin{verbatim}
          hw1 hw2 hw3 hw4 hw5
student-1 100  73 100  88  79
student-2  85  64  78  89  78
student-3  83  69  77 100  77
student-4  88  NA  73 100  76
student-5  88 100  75  86  79
student-6  89  78 100  89  77
\end{verbatim}

\begin{Shaded}
\begin{Highlighting}[]
\NormalTok{results }\OtherTok{\textless{}{-}} \FunctionTok{apply}\NormalTok{(gradebook, }\DecValTok{1}\NormalTok{, grade)}
\NormalTok{results}
\end{Highlighting}
\end{Shaded}

\begin{verbatim}
 student-1  student-2  student-3  student-4  student-5  student-6  student-7 
     91.75      82.50      84.25      84.25      88.25      89.00      94.00 
 student-8  student-9 student-10 student-11 student-12 student-13 student-14 
     93.75      87.75      79.00      86.00      91.75      92.25      87.75 
student-15 student-16 student-17 student-18 student-19 student-20 
     78.75      89.50      88.00      94.50      82.75      82.75 
\end{verbatim}

Q2. Using your grade() function and the supplied gradebook, Who is the
top scoring student overall in the gradebook? {[}3pts{]}

\begin{Shaded}
\begin{Highlighting}[]
\FunctionTok{max}\NormalTok{(results)}
\end{Highlighting}
\end{Shaded}

\begin{verbatim}
[1] 94.5
\end{verbatim}

\begin{Shaded}
\begin{Highlighting}[]
\FunctionTok{which.max}\NormalTok{(results)}
\end{Highlighting}
\end{Shaded}

\begin{verbatim}
student-18 
        18 
\end{verbatim}

Q3. From your analysis of the gradebook, which homework was toughest on
students (i.e.~obtained the lowest scores overall? {[}2pts{]}

\begin{Shaded}
\begin{Highlighting}[]
\FunctionTok{apply}\NormalTok{(gradebook, }\DecValTok{2}\NormalTok{, mean, }\AttributeTok{na.rm=}\NormalTok{T)}
\end{Highlighting}
\end{Shaded}

\begin{verbatim}
     hw1      hw2      hw3      hw4      hw5 
89.00000 80.88889 80.80000 89.63158 83.42105 
\end{verbatim}

\begin{Shaded}
\begin{Highlighting}[]
\FunctionTok{which.min}\NormalTok{( }\FunctionTok{apply}\NormalTok{(gradebook, }\DecValTok{2}\NormalTok{, mean, }\AttributeTok{na.rm=}\NormalTok{T))}
\end{Highlighting}
\end{Shaded}

\begin{verbatim}
hw3 
  3 
\end{verbatim}

\begin{Shaded}
\begin{Highlighting}[]
\FunctionTok{which.min}\NormalTok{( }\FunctionTok{apply}\NormalTok{(gradebook, }\DecValTok{2}\NormalTok{, sum, }\AttributeTok{na.rm=}\NormalTok{T))}
\end{Highlighting}
\end{Shaded}

\begin{verbatim}
hw2 
  2 
\end{verbatim}

Q4. Optional Extension: From your analysis of the gradebook, which
homework was most predictive of overall score (i.e.~highest correlation
with average grade score)? {[}1pt{]}

\begin{Shaded}
\begin{Highlighting}[]
\CommentTok{\# Make all (or mask) NA to zero}
\NormalTok{mask }\OtherTok{\textless{}{-}}\NormalTok{ gradebook}
\NormalTok{mask[ }\FunctionTok{is.na}\NormalTok{(mask) ] }\OtherTok{\textless{}{-}} \DecValTok{0}
\CommentTok{\#mask}
\end{Highlighting}
\end{Shaded}

We can use the \texttt{cor()} function for correlation analysis.

\begin{Shaded}
\begin{Highlighting}[]
\FunctionTok{cor}\NormalTok{(mask}\SpecialCharTok{$}\NormalTok{hw5, results)}
\end{Highlighting}
\end{Shaded}

\begin{verbatim}
[1] 0.6325982
\end{verbatim}

\begin{Shaded}
\begin{Highlighting}[]
\FunctionTok{cor}\NormalTok{(mask}\SpecialCharTok{$}\NormalTok{hw3, results)}
\end{Highlighting}
\end{Shaded}

\begin{verbatim}
[1] 0.3042561
\end{verbatim}

I need to use the \texttt{apply()} function to run this analysis over
the whole course (i.e.~masked gradebook)

\begin{Shaded}
\begin{Highlighting}[]
\FunctionTok{apply}\NormalTok{(mask, }\DecValTok{2}\NormalTok{, cor, results)}
\end{Highlighting}
\end{Shaded}

\begin{verbatim}
      hw1       hw2       hw3       hw4       hw5 
0.4250204 0.1767780 0.3042561 0.3810884 0.6325982 
\end{verbatim}

Q5. Make sure you save your Quarto document and can click the ``Render''
(or Rmarkdown''Knit'') button to generate a PDF format report without
errors. Finally, submit your PDF to gradescope. {[}1pt{]}



\end{document}
